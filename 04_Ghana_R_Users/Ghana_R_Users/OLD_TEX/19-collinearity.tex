Tests for Collinearity 

When there is a perfect linear relationship among the predictors, the estimates for a regression model cannot be uniquely computed. The term collinearity describes two variables are near perfect linear combinations of one another. When more than two variables are involved, it is often called multicollinearity, although the two terms are often used interchangeably. 

The primary concern is that as the degree of multicollinearity increases, the regression model estimates of the coefficients become unstable and the standard errors for the coefficients can get wildly inflated. In this section, we will explore some SAS options used with the model statement that help to detect multicollinearity. 

We can use the vif option to check for multicollinearity. vif stands for variance inflation factor. As a rule of thumb, a variable whose VIF values is greater than 10 may merit further investigation. Tolerance, defined as 1/VIF, is used by many researchers to check on the degree of collinearity. A tolerance value lower than 0.1 is comparable to a VIF of 10. It means that the variable could be considered as a linear combination of other independent variables. The tol option on the model statement gives us these values. Let's first look at the regression we did from the last section, the regression model predicting api00 from meals, ell and emer, and use the vif and tol options with the model statement.  

