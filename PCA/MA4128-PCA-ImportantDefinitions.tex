
\documentclass[a4paper,12pt]{article}
%%%%%%%%%%%%%%%%%%%%%%%%%%%%%%%%%%%%%%%%%%%%%%%%%%%%%%%%%%%%%%%%%%%%%%%%%%%%%%%%%%%%%%%%%%%%%%%%%%%%%%%%%%%%%%%%%%%%%%%%%%%%%%%%%%%%%%%%%%%%%%%%%%%%%%%%%%%%%%%%%%%%%%%%%%%%%%%%%%%%%%%%%%%%%%%%%%%%%%%%%%%%%%%%%%%%%%%%%%%%%%%%%%%%%%%%%%%%%%%%%%%%%%%%%%%%
\usepackage{eurosym}
\usepackage{vmargin}
\usepackage{amsmath}
\usepackage{graphics}
\usepackage{epsfig}
\usepackage{subfigure}
\usepackage{fancyhdr}
\usepackage{listings}
\usepackage{framed}
\usepackage{graphicx}

\setcounter{MaxMatrixCols}{10}
%TCIDATA{OutputFilter=LATEX.DLL}
%TCIDATA{Version=5.00.0.2570}
%TCIDATA{<META NAME="SaveForMode" CONTENT="1">}
%TCIDATA{LastRevised=Wednesday, February 23, 2011 13:24:34}
%TCIDATA{<META NAME="GraphicsSave" CONTENT="32">}
%TCIDATA{Language=American English}

\pagestyle{fancy}
\setmarginsrb{20mm}{0mm}{20mm}{25mm}{12mm}{11mm}{0mm}{11mm}
\lhead{MA4128} \rhead{Mr. Kevin O'Brien}
\chead{Principal Component Analysis}
%\input{tcilatex}

\begin{document}
	\tableofcontents
	%http://support.sas.com/publishing/pubcat/chaps/55129.pdf

\section{Review of Important Definitions}
\begin{itemize}
\item An observed variable can be measured directly, is sometimes called a measured variable or an indicator or a
manifest variable.
\item A principal component is a linear combination of weighted observed variables. Principal components are
uncorrelated and orthogonal.
\item A latent construct can be measured indirectly by determining its influence to responses on measured variables. A latent construct could is also referred to as a factor, underlying construct, or unobserved variable.
\item Factor scores are estimates of underlying latent constructs.
\item Unique factors refer to unreliability due to measurement error and variation in the data.
\item Principal component analysis minimizes the sum of the squared perpendicular distances to the axis of the
principal component while least squares regression minimizes the sum of the squared distances perpendicular to the
x axis (not perpendicular to the fitted line).
\item Principal component scores are actual scores.
\item Eigenvectors are the weights in a linear transformation when computing principal component scores.
Eigenvalues indicate the amount of variance explained by each principal component or each factor.
\item Orthogonal means at a 90 degree angle, perpendicular.
Obilque means other than a 90 degree angle.
\item An observed variable \textbf{\emph{loads}} on a factors if it is highly correlated with the factor, has an eigenvector of greater magnitude on that factor.
\item Communality is the variance in observed variables accounted for by a common factors. Communality is more
relevant to EFA than PCA.
\end{itemize}


\end{document}


