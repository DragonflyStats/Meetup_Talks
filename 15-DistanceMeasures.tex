

\documentclass[a4paper,12pt]{article}
%%%%%%%%%%%%%%%%%%%%%%%%%%%%%%%%%%%%%%%%%%%%%%%%%%%%%%%%%%%%%%%%%%%%%%%%%%%%%%%%%%%%%%%%%%%%%%%%%%%%%%%%%%%%%%%%%%%%%%%%%%%%%%%%%%%%%%%%%%%%%%%%%%%%%%%%%%%%%%%%%%%%%%%%%%%%%%%%%%%%%%%%%%%%%%%%%%%%%%%%%%%%%%%%%%%%%%%%%%%%%%%%%%%%%%%%%%%%%%%%%%%%%%%%%%%%
\usepackage{eurosym}
\usepackage{vmargin}
\usepackage{amsmath}
\usepackage{graphics}
\usepackage{epsfig}
\usepackage{framed}
\usepackage{subfigure}
\usepackage{fancyhdr}

\setcounter{MaxMatrixCols}{10}
%TCIDATA{OutputFilter=LATEX.DLL}
%TCIDATA{Version=5.00.0.2570}
%TCIDATA{<META NAME="SaveForMode"CONTENT="1">}
%TCIDATA{LastRevised=Wednesday, February 23, 201113:24:34}
%TCIDATA{<META NAME="GraphicsSave" CONTENT="32">}
%TCIDATA{Language=American English}

\pagestyle{fancy}
\setmarginsrb{20mm}{0mm}{20mm}{25mm}{12mm}{11mm}{0mm}{11mm}
\lhead{MA4128} \rhead{Kevin O'Brien} \chead{Distance Measures} %\input{tcilatex}


\begin{document}

\author{Kevin O'Brien}
\title{Cluster Analysis - Distance Measures and Proximity Matrices}


%http://www.econ.upf.edu/~michael/stanford/maeb4.pdf
%http://stn.spotfire.com/spotfire_client_help/hc/hc_distance_measures_overview.htm
\newpage
\section{Euclidean Distance}
The Euclidean distance between two points, x and y, with $k$ dimensions is calculated as:
\[ \sqrt{ \sum^{k}_{j=1} ( x_j - y_j)^2 } \]
The Euclidean distance is always greater than or equal to zero. The measurement would be zero for identical points and high for points that show little similarity.

\subsection{Example}
Compute the Euclidean Distance between the following points:
$X = \{1,5,4,3\}$ and $Y = \{2,1,8,7\}$

\begin{center}
\begin{tabular}{|c|c|c|c|}
  \hline
$x_j$	&	$y_j$	&   $x_j - y_j$	&	$(x_j - y_j)^2$	\\ \hline
1	&	2	&	-1	&	1	\\
5	&	1	&	4	&	16	\\
4	&	8	&	-4	&	16	\\
3	&	7	&	-4	&	16	\\ \hline
	&		&		&	49	\\ \hline
\end{tabular}
\end{center}
The Euclidean Distance between the two points is $\sqrt{49}$ i.e. 7.

\subsection{Squared Euclidean Distance}
The Squared Euclidean distance between two points, x and y, with $k$ dimensions is calculated as:
\[ \sum^{k}_{j=1} ( x_j - y_j)^2  \]
The Squared Euclidean distance may be preferred to the Euclidean distance as it is slightly less computational complex, without loss of any information.

%---------------------------------------------------------------------------%

\section{Standardized of Distance Measures}

Let us consider measuring the distances between between two points using
the three continuous variables pollution, depth and temperature. Let us suppose that a difference of 4.1 in terms of pollution is considered quite large and unusual, while a difference of 48 in terms of depth is large, but not particularly unusual.
What would happen if we applied the Euclidean distance formula to measure distance between two cases.
\begin{center}
\begin{tabular}{|c|c|c|}
  \hline
  % after \\: \hline or \cline{col1-col2} \cline{col3-col4} ...
Variables & case 1 & case 2 \\ \hline 
Pollution & 6.0 & 1.9 \\
Depth & 51 & 99 \\
Temp & 3.0 & 2.9 \\
  \hline
\end{tabular}
\end{center}

Here is the calculation for Euclidean Distance:
\[ d = \sqrt{(6.0 - 1.9)^2 + (51 - 99)^2 + (3.0 - 2.9)^2}   \]
 \[ d = \sqrt{16.81 + 2304 + 0.01} = \sqrt{2320.82} = 48.17 \]
\noindent The contribution of the second variable depth to this calculation is huge – one could say
that the distance is practically just the absolute difference in the depth values (equal to
|51-99| = 48) with only tiny additional contributions from pollution and temperature. These three variables are on
completely different scales of measurement and the larger depth values have larger differences, so they will dominate in the calculation of Euclidean distances.

\noindent The approach to take here is \textbf{standardization}, which is is necessary to balance out the contributions, and the
conventional way to do this is to transform the variables so they all have the same variance
of 1. At the same time we \textbf{\textit{center}} the variables at their means – this centering is not
necessary for calculating distance, but it makes the variables all have mean zero and thus
easier to compare. 

\noindent The transformation commonly called standardization is thus as follows:

\[\mbox{standardized value} = \frac{\mbox{observed value – mean}}{ \mbox{standard deviation}}\]
\begin{center}
\begin{tabular}{|c|c|c|c|c|c|c|}
  \hline
  % after \\: \hline or \cline{col1-col2} \cline{col3-col4} ...
Variables & Case 1 & Case 2 & Mean & Std. Dev & Case 1 (std) & Case 2 (std) \\ \hline
Pollution & 6.0 & 1.9 & 4.517	&	2.141	&	0.693	&	-1.222	\\
Depth & 51 & 99 & 74.433	&	15.615	&	-1.501	&	1.573	\\
Temp & 3.0 & 2.9 & 3.057	&	0.281	&	-0.201	&	-0.557	\\
  \hline
\end{tabular}
\end{center}

\[ d_{std} =  \sqrt{(0.693 - (- 1.222))^2 + (-1.501-1.573)^2 + (-0.201-(-0.557))^2} \]

\[ d_{std} = \sqrt{3.667 + 9.449 + 0.127} = \sqrt{13.243} = 3.639 \]

Pollution and temperature have higher contributions than before but depth still plays the
largest role in this particular example, even after standardization. But this contribution is
justified now, since it does show the biggest standardized difference between the samples. 
%--------------------------------------------------------------------------------------%
\newpage
\section{Manhattan (City Block) Distance}
The City block distance between two points, x and y, with $k$ dimensions is calculated as:
\[ \sum^{k}_{j=1} | x_j - y_j |  \]

The City block distance is always greater than or equal to zero. The measurement would be zero for identical points and high for points that show little similarity.

\subsection{Example}
Compute the Manhattan Distance between the following points: 
$X = \{1,3,4,2\}$ and $Y = \{5,2,5,2\}$


\begin{center}
\begin{tabular}{|c|c|c|c|}
  \hline
$x_j$	&	$y_j$	&   $x_j - y_j$	&	$| x_j - y_j |$	\\ \hline
1	&	5	&	-4	&	4	\\
3	&	2	&	1	&	1	\\
4	&	5	&	-1	&	1	\\
2	&	2	&	0	&	0	\\ \hline
& & & 6 \\
  \hline
\end{tabular}
\end{center}
The Manhattan Distance between the two points is 6.

\end{document}

%--------------------------------------------------------------------------------------%
\newpage
\section{Cluster Analysis : Proximity Matrices}

Using \textbf{\textit{nearest neighbour}} linkage, describe how the agglomeration schedule based on the following 
proximity matrix. With nearest neighbour, a case is assigned to the cluster of the case with which it has the shortest distance. Cluster are also joined on this basis.


% latex table generated in R 2.15.2 by xtable 1.7-1 package
% Tue May 14 19:17:33 2013
\begin{table}[ht]
\centering
\begin{tabular}{|r|rrrrrrrrrr|}
  \hline
Case & 1 & 2 & 3 & 4 & 5 & 6 & 7 & 8 & 9 & 10 \\
  \hline
1 & 0.00 & \textbf{4.82} & 89.39 & 85.97 & 46.26 & 71.87 & 56.42 & 23.75 & 31.57 & 11.70 \\
  2 & \textbf{4.82} & 0.00 & 94.24 & 38.96 & \textbf{5.55} & 35.07 & 74.52 & 71.27 & 61.84 & \textbf{4.84} \\
  3 & 89.39 & 94.24 & 0.00 & 57.65 & 27.27 & 25.31 & 20.89 & \textbf{2.84} & 63.50 & 89.39 \\
  4 & 85.97 & 38.96 & 57.65 & 0.00 & \textbf{22.94} & \textbf{7.13} & 70.49 & 23.09 & \textbf{12.75} & 85.97 \\
  5 & 46.26 & \textbf{5.55} & 27.27 & \textbf{22.94} & 0.00 & 39.44 & 17.43 & 79.22 & 14.47 & 46.26 \\
  6 & 71.87 & 35.07 & 25.31 & \textbf{7.13} & 39.44 & 0.00 & 27.50 & 30.65 & 13.34 & 71.87 \\
  7 & 56.42 & 74.52 & 20.89 & 70.49 & 17.43 & 27.50 & 0.00 & 91.16 & 44.92 & \textbf{6.42} \\
  8 & 23.75 & 71.27 & \textbf{2.84} & 23.09 & 79.22 & 30.65 & 91.16 & 0.00 & \textbf{3.18} & 23.75 \\
  9 & 31.57 & 61.84 & 63.50 & \textbf{12.75} & 14.47 & 13.34 & 44.92 & \textbf{3.18} & 0.00 & 31.57 \\
  10 & 11.70 & \textbf{4.84} & 89.39 & 85.97 & 46.26 & 71.87 & \textbf{6.42} & 23.75 & 31.57 & 0.00 \\
   \hline
\end{tabular}
\end{table}


\begin{itemize}
\item The closest pair in terms of distance (2.84) are cases 3 and 8. So this is the first linkage.
\item The next closest pair (3.18) are 8 and 9. The next linkage joins case 9 to 3 and 8.
\item The next closest pair (4.82) are 1 and 2. So this is the next linkage. [ So far (3,8,9) and (2,10) ] 
\item The next closest pair (4.84) are 2 and 10. The next linkage joins case 1 to 2 and 10.
\item The next closest pair (5.55) are 2 and 5. The next linkage joins case 5 to 1, 2 and 10. [ So far (3,8,9) and (1,2,5,10)]
\item The next closest pair (6.42) are 7 and 10. The next linkage joins case 7 to 1, 2, 5 and 10.
\item The next closest pair (7.13) are 4 and 6. The next linkage joins case 4 to 6. [ So far (3,8,9), (4,6) and (1,2,5,10) All cases are in clusters. This is a 3 cluster solution. ]
\item The next closest pair (11.70) are 1 and 10. Disregard, because they are already clustered together.
\item The next closest pair (19.44) are 4 and 9. This joins cluster (4,6) to cluster (3,8,9) [ So far (3,4,6,8,9) and (1,2,5,10). This is a 2 cluster solution.]
\item The next closest pairing is 4 and 5. This linkage joins all cases together in one cluster.
\end{itemize}




\end{document}