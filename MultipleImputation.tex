

\documentclass[a4paper,12pt]{article}
%%%%%%%%%%%%%%%%%%%%%%%%%%%%%%%%%%%%%%%%%%%%%%%%%%%%%%%%%%%%%%%%%%%%%%%%%%%%%%%%%%%%%%%%%%%%%%%%%%%%%%%%%%%%%%%%%%%%%%%%%%%%%%%%%%%%%%%%%%%%%%%%%%%%%%%%%%%%%%%%%%%%%%%%%%%%%%%%%%%%%%%%%%%%%%%%%%%%%%%%%%%%%%%%%%%%%%%%%%%%%%%%%%%%%%%%%%%%%%%%%%%%%%%%%%%%
\usepackage{eurosym}
\usepackage{vmargin}
\usepackage{amsmath}
\usepackage{graphics}
\usepackage{epsfig}
\usepackage{framed}
\usepackage{subfigure}
\usepackage{fancyhdr}

\setcounter{MaxMatrixCols}{10}
%TCIDATA{OutputFilter=LATEX.DLL}
%TCIDATA{Version=5.00.0.2570}
%TCIDATA{<META NAME="SaveForMode"CONTENT="1">}
%TCIDATA{LastRevised=Wednesday, February 23, 201113:24:34}
%TCIDATA{<META NAME="GraphicsSave" CONTENT="32">}
%TCIDATA{Language=American English}

\pagestyle{fancy}
\setmarginsrb{20mm}{0mm}{20mm}{25mm}{12mm}{11mm}{0mm}{11mm}
\lhead{MA4128} \rhead{Kevin O'Brien} \chead{Advanced Data Modelling} %\input{tcilatex}


\begin{document}

\subsection*{Multiple Imputation for Missing Data}
%(\textit{Source: https://www.statisticssolutions.com/multiple-imputation-for-missing-data/})
\smallskip
\noindent Multiple imputation for missing data is an attractive method for handling missing data in multivariate analysis. The idea of multiple imputation for missing data was first proposed by Rubin (1977).

\smallskip

Multiple imputation is a statistical technique for analyzing incomplete data sets, that is, data sets for which some entries are missing. Application of the technique requires three steps: imputation, analysis and pooling. 

\begin{figure}[h!]
\centering
\includegraphics[width=0.7\linewidth]{images/MI.png}
\end{figure}


\begin{description}
\item[Imputation:] Impute (=fill in) the missing entries of the incomplete data sets, not once, but m times (m=3 in the figure). Imputed values are drawn for a distribution (that can be different for each missing entry). This step results is $m$ complete data sets. 
\item[Analysis:] Analyze each of the m completed data sets. This step results in $m$ analyses.
\item[Pooling:] Integrate the m analysis results into a final result. Simple rules exist for combining the m analyses.
\end{description}
Rubin (1987) has shown that if the method to create imputations is 'proper', then the resulting inferences will be statistically valid.

\smallskip
\noindent  The most challenging step is Imputation, that is, the construction of the m completed data sets. This step accounts for the process that created the missing data. 

\smallskip
\noindent Typical problems are:
\begin{itemize}
    \item  the fact that something is missing could be related its value (e.g., people with higher incomes tend to skip income questions more often);
    \item  missing entries can appear anywhere in the data;
    \item  the method used in the imputation step must foresee the intended complete-data analyses.
\end{itemize}

\noindent The repeated analysis step on the imputed data is actually somewhat simpler than the same analysis without imputation, since there is no need to bother with the missing data.

\noindent The pooling step consists of computing the mean over the m repeated analysis, its variance, and its confidence interval or P value. In general, these computation are relatively simple.


\end{document}