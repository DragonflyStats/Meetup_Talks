
\documentclass[]{article}

\voffset=-1.5cm
\oddsidemargin=0.0cm
\textwidth = 480pt

\usepackage{framed}
\usepackage{subfiles}
\usepackage{graphics}
\usepackage{newlfont}
\usepackage{eurosym}
\usepackage{amsmath,amsthm,amsfonts}
\usepackage{amsmath}
\usepackage{enumerate}
\usepackage{color}
\usepackage{multicol}
\usepackage{amssymb}
\usepackage{multicol}
\usepackage[dvipsnames]{xcolor}
\usepackage{graphicx}
\begin{document}
\section*{Demonstration of Mean Imputation with R}
\begin{framed}
\begin{verbatim}

#### 1. Create a "True" Data set ####
TrueX <- rnorm(100,100,15)
TrueX <- round(TrueX)
TrueX


#### 2. Summary and Confidence Interval ####
summary(TrueX)

t.test(TrueX,mu=100)

#### 3. Create a data set with some value missing ####

# "part" for partial

index <- sample(1:100,10)
partX <- TrueX 
partX[index] <- NA

partX

#### 4. Summary and Confidence Interval ####

summary(partX)

t.test(partX,mu=100)

#### 5. Create an Imputation Value (mean or median) ####

# Wrong
imputeValue <- median(partX)

#Right
imputeValue <- mean(partX,na.rm=TRUE)

#### 6. Impute this value into the data ####

imputedX <- partX

imputedX [is.na(imputedX)] <- imputeValue

#### 7. Summary and Confidence Interval ####

summary(imputedX)

t.test(imputedX,mu=100)

\end{verbatim}
\end{framed}
\end{document}