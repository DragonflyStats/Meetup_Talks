

\documentclass[a4paper,12pt]{article}
%%%%%%%%%%%%%%%%%%%%%%%%%%%%%%%%%%%%%%%%%%%%%%%%%%%%%%%%%%%%%%%%%%%%%%%%%%%%%%%%%%%%%%%%%%%%%%%%%%%%%%%%%%%%%%%%%%%%%%%%%%%%%%%%%%%%%%%%%%%%%%%%%%%%%%%%%%%%%%%%%%%%%%%%%%%%%%%%%%%%%%%%%%%%%%%%%%%%%%%%%%%%%%%%%%%%%%%%%%%%%%%%%%%%%%%%%%%%%%%%%%%%%%%%%%%%
\usepackage{eurosym}
\usepackage{vmargin}
\usepackage{amsmath}
\usepackage{graphics}
\usepackage{epsfig}
\usepackage{framed}
\usepackage{subfigure}
\usepackage{fancyhdr}

\setcounter{MaxMatrixCols}{10}
%TCIDATA{OutputFilter=LATEX.DLL}
%TCIDATA{Version=5.00.0.2570}
%TCIDATA{<META NAME="SaveForMode"CONTENT="1">}
%TCIDATA{LastRevised=Wednesday, February 23, 201113:24:34}
%TCIDATA{<META NAME="GraphicsSave" CONTENT="32">}
%TCIDATA{Language=American English}

\pagestyle{fancy}
\setmarginsrb{20mm}{0mm}{20mm}{25mm}{12mm}{11mm}{0mm}{11mm}
\lhead{MA4128} \rhead{Kevin O'Brien} \chead{Advanced Data Modelling} %\input{tcilatex}


\begin{document}
\subsection*{Correlation Matrix of Predictor Variables}
\begin{figure}[h!]
\centering
\includegraphics[width=0.7\linewidth]{images/VIF1.png}
\end{figure}
\subsection*{Model Summary}
\begin{figure}[h!]
\centering
\includegraphics[width=0.7\linewidth]{images/VIF2.png}
\end{figure}

\subsection*{Critiquing The Model}
\begin{itemize}
    \item We see that the predictors Weight and BSA are highly correlated (r = 0.875). We can choose to remove either predictor from the model. The decision of which one to remove is often a scientific or practical one. For example, if the researchers here are interested in using their final model to predict the blood pressure of future individuals, their choice should be clear. 
    \item Which of the two measurements — body surface area or weight — do you think would be easier to obtain?! If indeed weight is an easier measurement to obtain than body surface area, then the researchers would be well-advised to remove BSA from the model and leave Weight in the model.

\item Reviewing again the above pairwise correlations, we see that the predictor Pulse also appears to exhibit fairly strong marginal correlations with several of the predictors, including Age (r = 0.619), Weight (r = 0.659) and Stress (r = 0.506). Therefore, the researchers could also consider removing the predictor Pulse from the model.
\item This should be carried out on a step-by-step basis. The aggregate result of the both steps together is presented below.
\end{itemize}

\newpage

\subsection*{Model Summary Where ``BSA" and ``Pulse" is omitted}

\begin{figure}[h!]
\centering
\includegraphics[width=0.7\linewidth]{images/VIF3.png}
\end{figure}


\end{document}